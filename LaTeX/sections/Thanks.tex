\section*{谢辞}
\addcontentsline{toc}{section}{谢辞}

从零开始实现一个操作系统,是我本科期间一次独特且宝贵的经历。这一路走来,实属不易。

实现操作系统的想法可以追溯到2024年寒假,当时我主要学习川合秀实所著的《30天自制操作系统》。然而,这本书所依赖的工具链过于陈旧,必须使用书中提供的编译器和特有的非标准函数,启动区更是为2.88MB软盘设计的。由于没有使用标准编译器,作为初学者的我必须严格按照教程操作,这让我在开发过程中感到束手束脚,无法灵活地增添新功能,这并不是我所追求的。

之后,我转而学习于渊所著的《Orange'S:一个操作系统的实现》,但因其难度过大,最终放弃。

4月,我在Bilibili平台偶然发现了一位宝藏UP主LunaixSky,他发布了一系列视频教程《从零开始自制操作系统》。这位作者的操作系统LunaixOS及相关教程完全原创,未参考任何现有的操作系统开发书籍或开源内核代码。为了编写这套教程,他耗费了大量时间和精力钻研技术文档和现行工业标准,力求知识的独创性。但随着教程的深入和项目的频繁重构,我在跟随教程到分页与内核重映射部分时便跟不上了。Git提交记录和视频教程的不完全对应加大了我的学习难度,让我最终不得不选择放弃。然而,这段学习经历让我受益匪浅,不仅提升了我的技术理解能力,还让我真正领悟到了开发操作系统的魅力,并培养了我查阅Intel®64与IA-32架构软件开发者手册等官方技术文档的能力。加入作者的LunaixOS技术交流群,更让我感受到了开源精神的伟大。

在期末周结束后的小学期,我重拾起这个项目。面对C语言中指针满天飞带来的复杂性和潜在的不安全因素,我做出了一个重要决定:将MinmusOS的编程语言从C语言改为Rust语言。Rust的开发体验相比C语言无疑更加出色。C语言中的许多不安全或未定义行为,往往只能在运行时暴露出问题或被隐藏,而Rust则通过其严格的安全性检查,在编译期就能有效避免这些错误。这种安全性和可靠性,极大地提升了我的开发效率。

自此,我不断在OS Dev Wiki学习并查阅Intel IA-32架构软件开发者手册,逐步完成了引导程序、内核、标准运行库与应用程序的实现。从最初学习和复现别人的代码,到逐渐成长为能够独立实现自定义功能的开发者,甚至为开源社区贡献代码、修复Bug,我感受到自己编程能力的巨大进步,也体会到了从无到有实现一个项目的成就感。回首整个项目,于我而言,这不仅仅是一项课程设计,更是我技术生涯中的重要里程碑。

在此,我要衷心感谢张惠娟老师和王冬青老师对于操作系统理论的教授,并让我有这个机会从零开始实现一个操作系统。感谢开源社区和OS Dev Wiki的贡献者们,是你们的无私分享让我不断增加对操作系统开发的理解。更要感谢一路坚持、没有因为技术难题而放弃的自己。最后,感谢阅读至此的你,愿你也能在自己的学习与探索中找到属于自己的光芒。

\hfill 2024年8月27日

\hfill 于同济大学四平路校区