\section{引导程序实现}\label{sec:BootloaderImplementation}

在本节(\cref{sec:BootloaderImplementation})中,

\subsection{引导过程概述}

引导加载器(Bootloader)是一种专门用于启动操作系统的程序。其主要职责是将操作系统的内核加载到内存中并执行。在此之前,引导加载器需要完成一系列准备工作,包括将CPU从16位实模式切换到32位保护模式,并对内存段进行配置。市场上存在多种成熟的引导加载器,例如GRUB,这类引导加载器能够启动Linux、Windows等复杂的操作系统。

但本项目采用了自制的引导加载器,全部使用Rust语言编写。本项目设计的引导加载器分为两个阶段。第一阶段的引导加载器由于必须适配磁盘的第一个扇区,因此在内存容量上极为受限。它的唯一目的是为了加载第二阶段的引导加载器。第一阶段的引导加载器功能简单,仅包含启动计算机和从磁盘读取第二阶段引导加载器所需的基本代码。一旦第一阶段的任务完成,控制权就会转交给第二阶段的引导加载器,后者负责更为复杂的任务,包括初始化额外的硬件设备、设置内存管理系统、将操作系统内核加载到内存中,并最终将控制权移交给内核。

这种分阶段的启动方式大大简化了各个阶段的职责,使每个部分都能集中处理其核心功能,避免过度复杂化。此外,采用Rust语言编写引导加载器增强了系统的安全性和可靠性。Rust语言内置的内存安全特性可以有效减少常见的内存错误,为整个引导过程提供了坚实的保障。

\subsection{第一阶段引导:实模式}

\subsubsection{主引导记录(Master Boot Record,MBR)}

主引导记录(MBR)是磁盘的第一个扇区(通常是扇区0),也被称为启动扇区,因为它包含了启动操作系统所需的引导程序(通常是引导加载程序的第一阶段),大小为512字节。MBR的结构(\cref{tab:MBRStructure})包括以下几个主要部分:

\begin{enumerate}
    \item \textbf{MBR 引导代码(Boot Code,0x000-0x1B7)}:包含用于启动计算机的基本输入输出系统(BIOS)的引导代码。在计算机启动时,BIOS 会加载并执行这段代码,这是启动序列的一部分。
    \item \textbf{磁盘唯一标识符(Disk Signature,0x1B8-0x1BB)}:用于存储磁盘的唯一标识符,这有助于操作系统识别和区分不同的磁盘。
    \item \textbf{保留区域(Reserved Area,0x1BC-0x1BD)}:通常未使用,保留供将来使用。
    \item \textbf{分区表条目(Partition Entries,0x1BE-0x1FD)}:存储关于硬盘上不同分区的信息,如起始地址、分区类型、分区大小等。每个分区表条目可以定义一个主分区或扩展分区。
    \item \textbf{标志性签名(Signature, 0x1FE-0x1FF)}:最后两个字节固定为 0x55AA,这是用来识别有效的引导扇区的标志。
\end{enumerate}

\begin{longtable}[c]{@{}ccc@{}}
    \caption{MBR结构}
    \label{tab:MBRStructure}                       \\
    \toprule
    \textbf{偏移量} & \textbf{大小 (字节)} & \textbf{描述}  \\ \midrule
    \endfirsthead
    \multicolumn{3}{r}{\textbf{续表~\thetable}}      \\
    \toprule
    \textbf{偏移量} & \textbf{大小 (字节)} & \textbf{描述}  \\ \midrule
    \endhead
    \hline
    \multicolumn{3}{r}{续下页}
    \endfoot
    \endlastfoot
    0x000        & 440              & MBR 引导代码     \\
    0x1B8        & 4                & 磁盘唯一标识符      \\
    0x1BC        & 2                & 保留区域         \\
    0x1BE        & 16               & 分区表条目        \\
    0x1CE        & 16               & 分区表条目        \\
    0x1DE        & 16               & 分区表条目        \\
    0x1EE        & 16               & 分区表条目        \\
    0x1FE        & 2                & 标志性签名 0x55AA \\ \bottomrule
\end{longtable}

当从磁盘引导时,BIOS自动将该磁盘的第一个扇区加载到内存地址0x7C00,并跳转到该地址执行MBR引导程序。

在MinmusOS中,MBR引导程序是引导加载器的第一阶段。这是一个极其受限的环境,因为必须只有440字节的程序能够将引导加载器的第二阶段加载到内存中并执行它。因此,这部分引导程序通常使用汇编语言编写,以实现尽可能的优化。然而,Rust编译器也能生成优化的二进制文件,使其适用于此目的。因此,MinmusOS的第一阶段引导程序主要用Rust编写,除了一小部分用汇编编写的程序,负责:

\begin{enumerate}
    \item 禁用硬件中断
    \item 将数据段寄存器置零
    \item 设置堆栈
    \item 调用Rust主函数
\end{enumerate}

这样的设计充分利用了Rust的性能优势,同时保留汇编语言处理底层硬件操作的能力。

\subsubsection{BIOS 中断(BIOS Interrupts)}

在引导加载器的第一阶段,由于内存环境非常受限,它无法实现自己的磁盘驱动程序。因此,引导加载器使用 BIOS 中断来访问硬件,执行各种任务,如在屏幕上打印信息或从磁盘读取数据。

BIOS 中断调用是由 BIOS 提供的函数,用于简化和抽象硬件访问。这些中断调用只能在 16 位实模式下工作,因此它们并不适合作为硬件驱动程序,而是仅用于在启动过程中帮助引导加载器工作。当 CPU 进入 32 位保护模式时,BIOS 中断就无法工作,因此内核必须实现自定义的硬件驱动程序。

MinmusOS 的引导加载器使用 BIOS 中断 0x13 从磁盘读取数据。这个中断要求设置一个磁盘地址包(Disk Address Packet,DAP)结构,用于指定要读取的扇区数、逻辑块地址(LBA),以及将数据写入内存的位置。数据结构定义如\cref{lst:DiskAddressPacketDataStructure}所示:

\begin{listing}[htbp]
    \begin{minted}{rust}
#[repr(C, packed)]
struct DiskAddressPacket {
    size: u8,
    zero: u8,
    sectors: u16,
    offset: u16,
    segment: u16,
    lba: u64,
}
    \end{minted}
    \caption{DiskAddressPacket数据结构定义}\label{lst:DiskAddressPacketDataStructure}
\end{listing}

\begin{enumerate}
    \item \texttt{size}:表示这个结构体的大小(以字节为单位)。用来向 BIOS 提供这个结构体大小的信息,确保 BIOS 正确解释剩余的字段。
    \item \texttt{zero}:这个字段通常设置为0,用于填充或确保结构体的对齐。
    \item \texttt{sectors}:指定要读取的扇区数量。因为 BIOS 中断 0x13 是以扇区为单位进行数据传输的,这个字段告诉 BIOS 一次操作需要读取多少扇区。
    \item \texttt{offset}:数据应该被加载到的内存段内的偏移地址。这告诉 BIOS 从磁盘读取数据后,数据应该存储在哪个具体的内存位置。
    \item \texttt{segment}:这是内存段的地址,与 offset 一起决定数据最终存放的物理内存位置。在实模式下,物理地址计算公式是 $segment \times 16 + offset$。
    \item \texttt{lba}:逻辑块寻址(Logical Block Addressing,LBA)的起始地址。这是一个64位的值,用来指定从哪个扇区开始读取数据。LBA 模式允许以线性方式访问硬盘上的扇区,而不需要考虑物理磁盘的几何结构(如柱面、磁头、扇区)。
\end{enumerate}

这个结构的打包(packed)属性是必须的,因为它确保编译器不会在成员之间插入填充,从而满足 BIOS 对于这种结构数据严格的内存布局要求。这是在低级系统编程中常见的做法,用以确保与硬件之间的接口按预期工作。

在发出中断之前,引导加载器需要设置一些 CPU 寄存器:

\begin{enumerate}
    \item \textbf{DS:SI}:设置为 DAP(磁盘地址包)的地址。DS 是段寄存器,SI 是偏移量寄存器,二者组合提供了数据结构的完整物理地址。
    \item \textbf{AH}:设置为 0x42。在 INT 0x13 的多个功能中,AH=0x42 对应于扩展读盘操作,它支持使用逻辑块地址(LBA)而非传统的柱面-磁头-扇区(CHS)寻址方式。
    \item \textbf{DL}:设置为驱动器号(对于主驱动器是 0x80)。DL 寄存器指定了要访问的磁盘驱动器号。
\end{enumerate}

然后发出 INT 0x13 中断将调用 BIOS 函数,从磁盘读取数据。如果在读取过程中发生错误,进位标志(carry flag)\footnote{进位标志(Carry Flag):这是 CPU 状态寄存器中的一位,用来指示上一个算术或逻辑操作是否产生了进位或借位。在使用 INT 0x13 时,如果操作成功,进位标志将被清除(设置为 0);如果操作失败,进位标志将被设置(设置为 1)。}将会被设置。MinmusOS 的引导加载器使用 JC 指令\footnote{JC(Jump if Carry)指令:这是一个条件跳转指令,只有在进位标志为 1 时才执行跳转。在引导加载器中,如果 INT 0x13 调用失败,进位标志会被设置,JC 指令将跳转到错误处理代码,通常会显示错误信息或停止进一步执行。}来检查这个标志,并通知用户错误发生。

另一个 BIOS 中断是 INT 0x10,INT 0x10 是一个 BIOS 提供的视频服务中断。这个中断提供了多种与显示相关的功能,如设置显示模式、更改字符颜色、移动光标以及打印字符到屏幕等,它在引导加载器中被用来打印字符串到屏幕上。然而,这个中断只在引导加载器中使用,因为内核实现了一个更复杂的打印功能,该功能直接写入到视频内存。

\subsubsection{启动模块(Boot Module)}

这个模块包括一个 Assembly 文件和其 Rust 接口的集成,负责禁用硬件中断,将数据段寄存器置零,设置堆栈,并且跳转到 Rust 主程序。

\begin{listing}[htbp]
    \begin{minted}{asm}
.section .boot, "awx"
.global _start
.code16

_start:
    cli

    xor ax, ax
    mov ds, ax
    mov es, ax
    mov ss, ax
    mov fs, ax
    mov gs, ax

    cld
    mov sp, 0x7c00

    call main

spin:
    hlt
    jmp spin
    \end{minted}
    \caption{boot/src/boot.asm}\label{lst:BootASM}
\end{listing}

这段汇编代码(\cref{lst:BootASM})是用于启动操作系统的Master Boot Record(MBR)的引导程序的一部分,位于硬盘的最开始的扇区。详细解释如下:

\begin{enumerate}
    \item \textbf{节和全局设置}
          \begin{enumerate}
              \item \texttt{.section .boot, "awx"}:定义一个名为 .boot 的段,属性为“awx”,表示该段是可分配的、可写的以及可执行的。
              \item \texttt{.global \_start}:定义 \_start 标签为全局,使得链接器可以找到它作为程序的入口点。
              \item \texttt{.code16}:指定代码为 16 位模式,适用于 BIOS 在实模式下工作。
          \end{enumerate}
    \item \textbf{禁用外部中断}
          \begin{enumerate}
              \item \texttt{cli}:Clear Interrupt Flag,清除中断标志,禁用硬件中断,确保在初始化过程中不会被外部事件打断。
          \end{enumerate}
    \item \textbf{设置数据段寄存器为零}
          \begin{enumerate}
              \item \texttt{xor ax, ax}:将 AX 寄存器清零。AX 寄存器是 x86 架构处理器中的一个通用寄存器,它是一个 16 位的寄存器,可以用于存储数据、执行算术和逻辑操作等。
              \item \texttt{mov ds, ax}等:将所有数据段寄存器(DS、ES、SS、FS、GS)设置为 0,初始化段寄存器,为后续程序的运行提供干净的段环境。
          \end{enumerate}
    \item \textbf{设置栈指针}
          \begin{enumerate}
              \item \texttt{cld}:Clear Direction Flag,清除方向标志,确保字符串操作在内存中从低地址向高地址移动。
              \item \texttt{mov sp, 0x7c00}:设置栈指针 SP 到 0x7c00,这是 BIOS 加载 MBR 程序的起始地址。由于栈是向下增长的,所以初始化栈指针到程序加载的起始地址是为了确保栈空间不会与程序空间冲突。
          \end{enumerate}
    \item \textbf{调用Rust主函数}
          \begin{enumerate}
              \item \texttt{call main}:调用标签为 main 的函数,这是用 Rust 编写的主函数,用于执行更复杂的任务。
          \end{enumerate}
    \item \textbf{防止程序执行溢出}
          \begin{enumerate}
              \item \texttt{hlt}:Halt 指令用于停止 CPU 的执行直到下一个外部中断被触发。
              \item \texttt{jmp spin}:无限循环,确保如果 main 函数返回,CPU 不会执行任何未定义的操作或跑到程序代码以外的地方去。
          \end{enumerate}
\end{enumerate}

在完成了启动模块的基础设置和功能调用之后,系统的硬件环境和内存状态被适当配置和初始化,为主启动程序的加载和执行提供了必要的条件。此时,CPU仍处于16位实模式,限制了对现代硬件特性的全面控制和访问。接下来,控制权将被传递到主启动程序,这一程序将负责进一步的系统启动过程,包括引导更复杂的操作系统核心组件。

\subsubsection{磁盘读取器模块(Disk Reader Module)}

这个模块负责从磁盘读取数据。它定义了 DiskReader 结构和相关方法,使用 BIOS 中断 0x13 来进行实际的磁盘操作。该模块使用了线性块地址(LBA)方式而不是传统的柱面-磁头-扇区(CHS)方式,这是现代磁盘访问的常用方法。这里使用的是一种称为磁盘地址包(Disk Address Packet,DAP)的数据结构(见\cref{lst:DiskAddressPacketDataStructure}),以支持这种访问方式。

DiskReader 结构体(\cref{lst:DiskReaderDataStructure})用于封装磁盘读取操作的状态,包括:

\begin{enumerate}
    \item \texttt{lba}:起始线性块地址
    \item \texttt{target}:数据应该加载到的内存地址
\end{enumerate}

\begin{listing}[htbp]
    \begin{minted}{rust}
pub struct DiskReader {
    lba: u64,
    target: u16,
}
    \end{minted}
    \caption{DiskReader数据结构定义}\label{lst:DiskReaderDataStructure}
\end{listing}

\paragraph{方法 \texttt{new(lba: u64, target: u16) -> Self}}

这个 new 方法是 DiskReader 结构体的构造函数,用于创建 DiskReader 实例。它接受两个参数:lba 和 target。lba 参数是一个 u64 类型的值,代表线性块地址,用来指定从哪个扇区开始读取数据。这允许构造函数支持大容量存储设备上的磁盘操作。target 参数是一个 u16 类型的值,表示数据加载到内存中的目标偏移地址。

通过这个构造函数,用户可以方便地创建一个配置好的 DiskReader 实例,随后可用于执行磁盘读取操作。这个设计简洁而有效,确保了 DiskReader 在创建时就被正确地配置。

\paragraph{方法 \texttt{read\_sector()}}

read\_sector 方法(\cref{lst:ReadSectorMethod})的主要功能是读取从特定线性块地址(LBA)开始的一个扇区的数据,并将数据存储到指定的内存偏移地址。该方法使用 DiskAddressPacket 结构体来配置读取操作的细节,并通过 INT 0x13 BIOS 中断进行磁盘访问。

\begin{listing}[htbp]
    \begin{minted}{rust}
pub fn read_sector(&self) {
    let dap = DiskAddressPacket {
        size: size_of::<DiskAddressPacket>() as u8,
        zero: 0,
        sectors: 1,
        offset: self.target,
        segment: 0x0000,
        lba: self.lba,
    };

    let dap_address = &dap as *const DiskAddressPacket;

    unsafe {
        core::arch::asm!(
        "mov {1:x}, si",
        "mov si, {0:x}",
        "int 0x13",
        "jc fail",
        "mov si, {1:x}",
        in(reg) dap_address as u16,
        out(reg) _,
        in("ax") 0x4200u16,
        in("dx") 0x0080u16,
        );
    }
}
    \end{minted}
    \caption{\texttt{read\_sector()}方法}\label{lst:ReadSectorMethod}
\end{listing}

在这个方法中,首先创建了一个 DiskAddressPacket 的实例 dap,具体字段配置如下:

\begin{enumerate}
    \item \texttt{size}:结构体的大小,使用 \texttt{size\_of::<DiskAddressPacket>()} 动态获取,保证与实际定义匹配。
    \item \texttt{zero}:固定为0,作为填充字段。
    \item \texttt{sectors}:设置为1,表示此次操作只读取一个扇区。
    \item \texttt{offset}和\texttt{segment}:指定数据加载到内存中的位置。offset 为 self.target,是调用时指定的内存地址偏移;segment 设置为 0x0000,通常是在实模式下使用的段基址。
    \item \texttt{lba}:从该线性块地址读取数据,对应于 DiskReader 实例中存储的 lba 字段。
\end{enumerate}

在 DiskAddressPacket 配置完成后,代码获取这个结构体的地址 dap\_address,然后通过内联汇编进行以下操作:

\begin{enumerate}
    \item \textbf{保存和设置寄存器}
          \begin{enumerate}
              \item \texttt{mov {1:x}, si}:保存原始 si 寄存器的值,以便恢复。
              \item \texttt{mov si, {0:x}}:将 DiskAddressPacket 的地址放入 si 寄存器,因为 INT 0x13 需要通过 si 传递参数。
          \end{enumerate}
    \item \textbf{执行 BIOS 中断}
          \begin{enumerate}
              \item \texttt{int 0x13}:执行磁盘读操作。ax 寄存器设置为 0x4200,表示执行读操作;dx 设置为 0x0080,表示第一个硬盘。
          \end{enumerate}
    \item \textbf{错误检查}
          \begin{enumerate}
              \item \texttt{jc fail}:如果操作失败(进位标志被设置),则跳转到错误处理标签 fail(需要在代码中定义该标签的行为)。
          \end{enumerate}
    \item \textbf{恢复寄存器}
          \begin{enumerate}
              \item \texttt{mov si, {1:x}}:恢复 si 寄存器的原始值。
          \end{enumerate}
\end{enumerate}

这种方式直接使用 BIOS 提供的功能来访问硬件,允许在没有操作系统支持的环境下(如引导加载器)进行低层次的磁盘操作。通过这个方法,DiskReader 可以读取磁盘上特定位置的数据,对于操作系统的启动和运行至关重要。

\paragraph{方法 \texttt{read\_sectors(sectors: u16)}}

该 read\_sectors 方法(\cref{lst:ReadSectorsMethod})是 DiskReader 结构体的成员方法,其功能是读取多个连续的扇区数据。方法接受一个参数 sectors,表示需要读取的扇区数量。该方法通过循环调用 read\_sector 方法来逐一读取指定数量的扇区,并适当更新目标内存地址和LBA地址。

\begin{listing}[htbp]
    \begin{minted}{rust}
pub fn read_sectors(&mut self, sectors: u16) {
    let mut sectors_left = sectors;
    while sectors_left > 0 {
        self.read_sector();
        self.target += SECTOR_SIZE;
        self.lba += 1;
        sectors_left -= 1;
    }
}
    \end{minted}
    \caption{\texttt{read\_sectors(sectors: u16)}方法}\label{lst:ReadSectorsMethod}
\end{listing}

以下是对该方法的解释:

\begin{enumerate}
    \item \textbf{初始化计数器}:创建一个变量 sectors\_left 用于追踪还剩多少扇区需要读取。
    \item \textbf{循环读取扇区}:使用一个循环来连续读取扇区,直到 sectors\_left 减到0,这意味着所有指定的扇区都已经读取完成。
    \item \textbf{读取单个扇区}:调用 read\_sector 方法读取一个扇区的数据。
    \item \textbf{更新内存地址和LBA地址}:每读取一个扇区后,将目标内存地址向前移动一个扇区的大小(512字节),这样下一个扇区的数据就不会覆盖前一个扇区的数据;将LBA地址递增1,以便下一次读取操作定位到下一个扇区。
    \item \textbf{递减剩余扇区计数}:每次循环结束后,将剩余扇区数减1。
\end{enumerate}

这个方法通过更新内存偏移地址和LBA地址,使得可以连续地读取多个扇区到指定的内存区域中。

\subsubsection{主启动程序(Main Boot Program)}

这个模块(\cref{lst:BootSrcMain})是系统的主入口,使用 Rust 编写。它负责加载并执行 bootloader,并处理初始化中的错误。

\begin{listing}[htbp]
    \begin{minted}{rust}
#![no_std]
#![no_main]

...

const BOOTLOADER_LBA: u64 = 2048;
const BOOTLOADER_SIZE: u16 = 64;

global_asm!(include_str!("boot.asm"));

extern "C" { static _bootloader_start: u16; }

#[no_mangle]
pub extern "C" fn main() -> ! {
    unsafe { asm!("mov ah, 0x00", "mov al, 0x03", "int 0x10"); }
    print("[INFO] Boot MinmusOS...\r\n\0");
    print("[INFO] Loading Bootloader...\r\n\0");
    let bootloader_start: *const u16 = unsafe { &_bootloader_start };
    let target = bootloader_start as u16;
    let mut disk = DiskReader::new(BOOTLOADER_LBA, target);
    disk.read_sectors(BOOTLOADER_SIZE);
    unsafe { asm!("jmp {0:x}", in(reg) bootloader_start as u16); }
    loop {}
}

fn print(message: &str) {
    unsafe {
        asm!(
        "mov si, {0:x}", // 将 message 的地址加载到 SI 寄存器
        "2:",            // 标签 2,这是一个循环的开始点
        "lodsb",         // 从 SI 指向的地址加载一个字节到 AL 寄存器,并将 SI 自增
        "or al, al",     // 将 AL 寄存器的内容与其自身进行 OR 操作,主要用来设置零标志(ZF)
        "jz 3f",         // 如果结果为零(即 AL 为零,表明字符串结束),则跳转到标签 3
        "mov ah, 0x0e",  // 将 0x0E 加载到 AH 寄存器,设置 BIOS 的 teletype 输出
        "mov bh, 0",     // 将显示页面设置为 0,通常用于多页面管理
        "out 0xe9, al",  // 输出到 0xE9 端口,这通常是用于调试的端口
        "int 0x10",      // 调用 BIOS 视频中断,使用 AH = 0x0E 和 AL 的值在屏幕上打印字符
        "jmp 2b",        // 无条件跳回到标签 2,继续循环读取下一个字符
        "3:",            // 标签 3,循环结束的地方
        in(reg) message.as_ptr(),
        );
    }
}

#[no_mangle]
pub extern "C" fn fail() -> ! {
    print("[ERROR] Failed to Load Bootloader!\r\n\0");
    loop {}
}

#[panic_handler]
fn panic(_info: &PanicInfo) -> ! { loop {} }
    \end{minted}
    \caption{boot/src/main.rs}\label{lst:BootSrcMain}
\end{listing}

\paragraph{全局和外部声明}

\texttt{\#![no\_std]}:指示编译器不使用标准库,这是操作系统和裸机程序的常见需求,因为标准库依赖于操作系统的功能。

\texttt{\#![no\_main]}:表示没有常规的 main 函数入口,这是裸机或操作系统开发中常见的设置。

\texttt{BOOTLOADER\_LBA}:定义引导加载器所在的逻辑块地址(LBA)。

\texttt{BOOTLOADER\_SIZE}:定义引导加载器大小,单位为扇区。

\texttt{global\_asm!}:插入全局汇编代码,这里包括从“boot.asm”文件中读取的汇编代码,通常用于设置初步的启动环境。

\paragraph{主入口函数 \texttt{main}}

main 函数是 MinmusOS 引导程序的核心入口点,它负责初始化系统并引导操作系统。在裸机或操作系统开发中,这个函数代替了常规应用程序中的标准 main 函数。\texttt{\#[no\_mangle]} 属性用于防止编译器修改函数名的装饰或混淆,即确保函数名在编译后保持不变。以下是对这个函数中每一步操作的详细解释:

\begin{enumerate}
    \item \textbf{设置视频模式}:\texttt{mov ah, 0x00} 和 \texttt{mov al, 0x03} 这两条指令准备寄存器 ah 和 al 以设置 BIOS 视频中断(int 0x10)的参数。ah = 0x00 代表设置视频模式的功能,al = 0x03 指定具体的视频模式,这里是标准的文本模式(80 \times 25 字符)。
    \item \textbf{打印启动信息}:调用 print 函数来在屏幕上显示启动信息,帮助用户了解当前引导进程的状态。
    \item \textbf{获取引导加载器起始地址}:通过外部链接的静态变量 \_bootloader\_start(由链接器脚本定义)获取引导加载器的起始内存地址。
    \item \textbf{初始化磁盘读取器}:将引导加载器的起始地址转换为 u16 类型,以便作为磁盘读取的目标内存地址。然后创建一个 DiskReader 实例,用于从指定的 LBA 地址开始读取数据到内存。
    \item \textbf{读取引导加载器到内存}:调用 DiskReader 的 read\_sectors 方法从磁盘读取 BOOTLOADER\_SIZE 个扇区的数据到内存中 target 指定的位置。
    \item \textbf{跳转到引导加载器执行}:使用内联汇编指令 jmp 直接跳转到引导加载器的起始地址,开始执行引导加载器的代码。
    \item \textbf{无限循环}:在正常情况下,jmp 指令后的代码不应该被执行,因为控制权已经转移给引导加载器。无限循环确保了在任何意外情况下,CPU 不会执行未定义的内存区域。
\end{enumerate}

\paragraph{打印函数 \texttt{print}}

在 MinmusOS 中的 print 函数负责将文本消息输出到屏幕,它通过 BIOS 中断 int 0x10 实现。这个函数使用内联汇编来直接与硬件交互。

\paragraph{错误处理函数 \texttt{fail}}

错误处理函数 fail 在 MinmusOS 的启动过程中加载引导加载器失败时被调用,用于显示错误信息并将系统置于无限循环状态,从而防止执行进一步的可能错误操作。

\paragraph{紧急停止处理函数 \texttt{panic}}

紧急停止处理(panic handler)是一种特殊的函数,用于处理运行时遇到的不可恢复的错误。它在程序遇到严重错误时被调用,如内存访问错误、预期之外的执行分支等。该函数的主要目的是防止程序继续执行可能危险或未定义的操作,它通过进入一个无限循环,使系统停留在已知的状态,便于调试和维护。在实际部署中,这种机制对于保持系统的稳定性和安全性至关重要,尤其是在裸机环境或操作系统的底层实现中。

\subsubsection{链接器脚本(Linker Script)}

链接器脚本(\cref{lst:BootLinkerScript})用于定义和控制操作系统引导程序的内存布局。脚本总体可以分为以下几个关键部分来概括其作用:

\begin{enumerate}
    \item \textbf{入口点配置}:脚本指定了程序的入口点 \_start,确保程序加载后从正确的位置开始执行。
    \item \textbf{栈配置}:脚本设置了栈的起始和结束位置,确保程序在执行时具有正确的栈空间进行操作。
    \item \textbf{段定义}:脚本详细定义了多个段,包括引导代码段 .boot,程序的主要执行代码段 .text,只读数据段 .rodata,以及包含初始化的全局变量和静态变量的数据段 .data。这些段的配置关键支持了程序数据和代码的正确加载和执行。
    \item \textbf{磁盘和分区表配置}:脚本设置了磁盘标识符和两个分区表,详细定义了启动分区和主分区的属性,这对于系统启动和磁盘管理至关重要。
    \item \textbf{魔数和引导加载器位置}:脚本的最后部分定义了引导扇区的魔数 0xAA55,保证引导扇区的有效性,同时标记了引导加载器的起始位置,确保引导过程可以正确地定位和加载引导加载器。
\end{enumerate}

整个链接器脚本确保了引导程序的内存布局和磁盘布局的准确性,对于系统的引导和加载过程至关重要。

\begin{listing}[htbp]
    \begin{minted}{text}
/* 配置程序入口点 */
ENTRY(_start)

/* 配置段的顺序和位置 */
SECTIONS {
    /* 配置栈的开始和结束位置 */
    . = 0X500;
    _stack_start = .;
    . = 0X7C00;
    _stack_end = .;

    /* 定义引导代码段,包含所有 .boot 和 .boot.* 段的内容 */
    .boot : { ... }

    /* 定义文本段,包含程序的主要执行代码 */
    .text : { ... }

    /* 定义只读数据段,包含字符串常量等不可修改的数据 */
    .rodata : { ... }

    /* 定义数据段,包含初始化的全局变量和静态变量 */
    .data : { ... }

    /* 调整内存布局中当前段的起始地址 */
    . = 0X7C00 + 0X1B8;

    /* 定义磁盘唯一标识符 */
    .diskid : { ... }

    /* 保留区域,设置为 0 */
    .reserved : { ... }

    /* 第一个分区表:用于存储引导加载器 */
    .first_table : { ... }

    /* 第二个分区表:用于主分区 */
    .second_table : { ... }

    /* 定义设置魔数位置,确保位于扇区的最后两个字节 */
    . = 0X7C00 + 0X1FE;

    /* 定义引导扇区有效的结束标记,必须是 0XAA55 */
    .magic_number : { ... }

    /* 定义引导加载器的起始位置标记 */
    _bootloader_start = .;
}
    \end{minted}
    \caption{boot/linker.ld}\label{lst:BootLinkerScript}
\end{listing}

在 Rust 项目中,build.rs 文件充当构建脚本的角色,主要用于在项目编译前执行自定义的构建任务。此文件对于那些需要细粒度控制编译过程的项目尤为重要,如操作系统开发或需要特定编译器设置的应用程序。

\begin{listing}[htbp]
    \begin{minted}{rust}
use std::path::Path;

fn main() {
    let local_path = Path::new(env!("CARGO_MANIFEST_DIR"));
    println!("cargo:rustc-link-arg-bins=--script={}", local_path.join("linker.ld").display());
}
    \end{minted}
    \caption{boot/build.rs}\label{lst:BootBuildRust}
\end{listing}

脚本使用 \texttt{env!("CARGO\_MANIFEST\_DIR")} 来获取环境变量,该变量指向包含 Cargo.toml 文件的目录,即项目的根目录。通过打印特定格式的字符串动态地向 Cargo 传递编译参数。\cref{lst:BootBuildRust}配置了编译器在构建二进制文件时使用自定义的链接器脚本。这通过 \texttt{println!} 输出 \texttt{cargo:rustc-link-arg-bins=--script={}} 实现,其中包含了链接器脚本的路径。通过计算出链接器脚本 linker.ld 的完整路径,并将其传递给编译器作为参数,build.rs 确保了链接过程按照预定义的内存布局和符号解析规则进行。这对于需要精确控制输出二进制格式的低级应用程序(如操作系统内核)是必需的。

\subsection{第二阶段引导:保护模式与内核加载}