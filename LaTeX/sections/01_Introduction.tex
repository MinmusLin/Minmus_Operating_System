\section{引言}\label{sec:Introduction}

在本节(\cref{sec:Introduction})中,

\subsection{项目背景}

\subsection{项目简介}

\subsection{基本概念}

\subsubsection{操作系统(Operating System)}

操作系统是一种控制计算机系统及其资源运行的软件。它是许多计算设备的核心软件组件,包括个人计算机、服务器、智能手机和嵌入式设备。操作系统通过提供多个硬件抽象层,使用户和应用程序能够通过一个简单的接口使用设备,隐藏了底层硬件的复杂性。根据操作系统的复杂程度,它们可以满足不同的需求。例如,多任务操作系统,如本项目设计与实现的系统,可以在同一台机器上同时运行多个任务。

操作系统的主要功能可能包括:

\begin{enumerate}
    \item 管理内存和其他系统资源
    \item 实施安全和访问策略
    \item 调度和复用进程和线程
    \item 动态启动和关闭用户程序
    \item 提供基本的用户界面和应用程序编程接口
\end{enumerate}

这意味着操作系统承担了以有组织、优化和安全的方式管理可用资源的责任。当进程运行时,操作系统必须确保不会发生意外和有害的行为,例如进程读取另一个进程的内存或占用CPU过长时间。正因如此,操作系统必须尽可能安全。而Rust编程语言可以通过其编译器在编译时保证内存安全和线程安全,从而满足这一要求。

并非所有的操作系统都提供所有这些功能。例如,单任务系统如MS-DOS不会调度进程,而嵌入式系统如eCOS可能没有用户界面,或者只能与一组静态的用户程序一起工作。

操作系统不是:

\begin{enumerate}
    \item 计算机硬件
    \item 特定的应用程序(如文字处理器、网页浏览器等)
    \item 一套实用程序(如在许多Unix派生系统中使用的GNU工具)
    \item 开发环境(尽管一些操作系统,如UCSD Pascal或Smalltalk-80,包含解释器和IDE)
    \item 图形用户界面(尽管许多现代操作系统将GUI作为操作系统的一部分)
\end{enumerate}

尽管大多数操作系统都会配备这些工具,但它们本身并不是操作系统的必要部分。一些操作系统,如Linux,可能以几种不同的打包形式分发,这些形式被称为发行版,它们可能拥有不同的应用程序和实用程序套件,并可能以不同的方式组织系统的某些方面。尽管如此,它们都是同一基本操作系统的版本,不应被视为不同类型的操作系统。

\subsubsection{内核(Kernel)}

内核是操作系统的核心部分,它负责处理所有基础级别的任务,尤其是与硬件直接交互的任务。内核的角色是桥接硬件和运行在计算机上的应用程序之间的通信。用户通常不会直接与内核交互,但它是系统运行的基础。

内核的主要职责包括:

\begin{enumerate}
    \item \textbf{中断处理}:内核响应由硬件产生的事件(称为中断)。例如,当按下键盘上的键时,内核的中断处理程序会获取按键编号,并将其转换成相应的字符存储在缓冲区中,供其他程序使用。
    \item \textbf{系统调用}:系统调用是用户级程序请求操作系统服务的方式。例如,打开文件、启动其他程序等。内核需要检查传入的参数是否有效,然后执行内部操作以完成请求。
    \item \textbf{资源管理}:内核管理计算机的资源,如CPU、内存和输入/输出设备。它负责资源的分配和回收,确保系统的高效运行。
    \item \textbf{进程调度}:内核负责进程和线程的调度。它决定哪个程序在何时使用CPU,以及如何在多个程序间公平地分配处理器时间。
    \item \textbf{抽象层}:内核通常定义一些基础的抽象概念,如文件、进程、套接字和目录等,这些都对应于它记住的关于上次操作的内部状态,使程序可以更高效地执行一系列操作。
\end{enumerate}

用户程序通常不直接发出系统调用(除了某些汇编程序),而是使用标准库来处理与内核交互所需的参数格式化和系统调用的生成。

总之,内核是操作系统最关键的组成部分,它提供了一个稳定和安全的环境,使得上层的应用程序能够不用关心硬件的复杂性而进行运行。

\subsubsection{命令行解释器(Shell)}

Shell是一个特殊的程序,通常集成在操作系统发行版中,为用户提供与计算机交互的界面。Shell的表现形式可能因系统而异(命令行、文件浏览器等),但其基本概念始终如一:

\begin{enumerate}
    \item \textbf{程序启动}:允许用户选择要启动的程序,并可选地给它提供特定于会话的参数。
    \item \textbf{文件操作}:允许对本地存储执行简单操作,如列出目录内容、在系统中移动和复制文件等。
\end{enumerate}

为了完成这些操作,Shell需要发出多个系统调用。Shell也可被其他程序用来启动程序。

现代的Shell还具有各种额外功能,包括:

\begin{enumerate}
    \item \textbf{自动补全}:通过按TAB键(或任何首选键),正在输入的单词将被自动补全为有效的Shell命令、文件、目录等。多次按自动完成键可以在其他补全可能性之间切换。
    \item \textbf{字符插入}:用户可以使用箭头键在输入的内容中移动。在句子中间键入新字符时,字符将被“插入”。
    \item \textbf{Shell历史记录}:通过使用上下箭头键,用户可以浏览之前的输入。
    \item \textbf{滚动}:当控制台的行数超过其高度时,将输出保存在缓冲区中,并允许用户在控制台中上下滚动。
    \item \textbf{脚本}:一些Shell具有自定义的脚本语言。脚本语言的例子包括Bash或DOS批处理。
\end{enumerate}

总之,Shell是操作系统中的一个核心组件,它通过友好的用户界面简化了用户与系统的交互,同时提供了执行复杂任务的强大命令行工具。

\subsubsection{英特尔IA-32架构(Intel IA-32 Architecture)}

Intel IA-32架构,也被称为x86,是Intel开发的一种32位指令集架构。从90年代后期开始,IA-32架构成为了广泛计算设备的基础,尤其是在个人计算机和服务器领域。它的发展重点之一是向后兼容性,使其能够运行几十年来开发的大量软件。这一特性使得IA-32架构在计算领域中发挥了重要作用,成为了行业标准。

IA-32是一种复杂指令集计算机(CISC)架构,允许在单条指令内执行复杂的操作。它包含多个通用寄存器,支持多种数据类型和寻址模式。此外,IA-32架构还引入了保护模式(Protected Mode),这是一个操作模式,提供了硬件级别的安全功能,如特权级别、分段、虚拟内存和分页。这些特性使得IA-32架构不仅具有强大的性能,还为操作系统提供了增强的安全性和稳定性。

\subsubsection{实模式(Real Mode)}

实模式(Real Mode)是一种处理器运行模式,最早由Intel 8086 CPU提出。它是最简单的CPU运行模式,也是所有x86处理器在启动时的默认模式。实模式主要特征如下:

\begin{enumerate}
    \item \textbf{内存寻址能力}:在实模式下,处理器只能直接寻址1MB( $2^{20}$ 字节)的内存。这是因为地址线只有20根,地址寄存器和段寄存器组合起来生成一个20位的物理地址。
    \item \textbf{无内存保护}:实模式不支持现代操作系统常用的多任务处理和内存保护机制。所有程序都对全内存有完全的访问权限,没有任何隔离机制,因此一个程序可以轻易地影响到系统中的其他程序。
    \item \textbf{兼容性}:实模式提供了与早期的8086和8088处理器的向后兼容性,允许旧的软件在新处理器上无修改地运行。
    \item \textbf{段寄存器}:使用段寄存器(如CS、DS、ES、SS)来引用内存。这些寄存器存储一个16位的段地址,通过与偏移量相结合,形成一个完整的物理地址。例如,段地址放在段寄存器,偏移量放在其他如IP或DI等寄存器中,二者结合才能访问到具体的内存位置。
\end{enumerate}

实模式的限制导致了后来保护模式(Protected Mode)的发展,保护模式支持更大的内存寻址空间(最初是16MB,后来扩展到更多),并引入了内存保护和多任务等特性,是现代操作系统所依赖的运行模式。

\subsubsection{保护模式(Protected Mode)}

保护模式(Protected Mode)是x86架构处理器支持的一种运行模式,首次出现在Intel 80286微处理器中。这种模式提供了比实模式更为高级的功能,包括更大的内存寻址能力、多任务支持以及严格的内存保护机制。保护模式的主要特点包括:

\begin{enumerate}
    \item \textbf{扩展的内存寻址}:在保护模式下,处理器可以寻址超过1MB的内存空间。例如,80286可以寻址高达16MB的内存,而更高级的80386及以后的处理器可以寻址高达4GB的内存。
    \item \textbf{内存保护}:保护模式引入了内存保护机制,每个程序都在其自己的内存空间运行,互不干扰。操作系统可以控制程序对内存的访问权限,防止程序访问非授权的内存区域。
    \item \textbf{多任务支持}:保护模式支持硬件级的多任务处理。处理器可以在不同的程序之间快速切换,每个程序运行在独立的环境中,提高了系统的效率和稳定性。
    \item \textbf{分段和分页}:在保护模式下,内存依旧被分为不同的段,但每个段的属性(如段的大小、访问权限等)可以通过段描述符在全局描述符表(GDT)或局部描述符表(LDT)中定义。此外,保护模式还支持分页机制,这允许操作系统将物理内存分成固定大小的页面,实现虚拟内存的管理,这样程序可以使用比物理内存更多的虚拟内存地址。
    \item \textbf{中断和异常处理}:保护模式增强了对中断和异常的处理能力。每种中断和异常都可以通过中断描述符表(IDT)进行配置,使得系统能更有效地响应各种硬件和软件事件。
\end{enumerate}

保护模式的引入标志着从单一的、不受保护的计算环境向现代多任务操作系统的重大转变。这种模式是现代操作系统如Windows、Linux等能够实现高效、稳定运行的基础。

\subsubsection{Rust语言}

Rust是一种系统级编程语言,近年来逐渐成为操作系统开发的重要工具。与大多数操作系统通常使用的C语言相比,Rust在内存管理和安全性方面引入了全新的方法。

C语言因其对硬件的低级别访问能力、直接内存操作的特性以及高效代码编写的能力而广泛用于操作系统开发。然而,这种低级别的控制权也带来了相应的风险。程序员必须手动管理内存,包括显式地分配和释放内存。这种灵活性虽然强大,但如果处理不当,就会引发严重的漏洞或问题,例如内存泄漏和悬空指针(dangling pointers)。这些问题不仅难以调试,还可能被恶意攻击者利用,导致执行任意代码或泄露敏感用户数据。

为了避免这些问题,高级编程语言通常使用垃圾回收(Garbage Collection)技术,通过运行时环境自动管理内存。当程序运行时,垃圾回收器会定期暂停程序,扫描内存以寻找不再使用的变量并自动释放它们。然而,这种方法的缺点是会带来性能开销和长时间的暂停,因此这些语言通常不用于操作系统的开发。

Rust编程语言采用了一种完全不同的内存管理方法,利用其独特的所有权(Ownership)机制在编译时确保内存操作的安全性,从而在运行时几乎不带来额外的开销。Rust不仅保持了类似于C和C++的低级别编程语言的性能,还提供了高级编程语言的内存安全性,结合了两者的优势而没有明显的缺点。

所有权机制的概念实际上非常直观,它遵循一组简单的规则来管理内存:

\begin{enumerate}
    \item 每个值都有一个所有者
    \item 同一时间只能有一个所有者
    \item 当所有者超出作用域时,该值会被自动释放
\end{enumerate}

这些规则使得Rust能够在保持与C和C++类似的灵活性的同时,确保内存的安全性。Rust的设计理念使其成为一个既能提供系统级语言性能,又能在编译时防止内存错误的强大工具,特别适合用于操作系统开发和其他需要高性能和安全性的应用程序。

\subsection{文档结构}